\documentclass[ngerman,xcolor={table,dvipsnames},smaller,compress,hyperref={bookmarks,colorlinks}]{beamer}

\usepackage{url}
\usepackage{listings}
\usepackage[latin9]{inputenc}
\usepackage{xcolor,textcomp}
%\usepackage{auto-pst-pdf}
\usepackage{pstricks}
\usepackage{pst-node}
\usepackage{pst-uml}
\usepackage{tabularx,threeparttable}
\usepackage{hyperref}

\title{Unix::Statgrab - System Monitoring}
\author{Jens Rehsack}
\date{2013}

\usetheme[secheader]{Boadilla}
\setbeamertemplate{navigation symbols}{}

\newcommand{\perlfilename}[1]{{\color{cyan}{\textit{\begingroup \urlstyle{sf}\Url{#1}}}}}
\newcommand{\xsfilename}[1]{{\color{olive}{\textit{\begingroup \urlstyle{sf}\Url{#1}}}}}
\newcommand{\hfilename}[1]{{\color{olive}{\textit{\begingroup \urlstyle{sf}\Url{#1}}}}}
\newcommand{\cfilename}[1]{{\color{magenta}{\textit{\begingroup \urlstyle{sf}\Url{#1}}}}}

\makeatletter
\makeatother

%\newcommand{\mypart}[1]{\part{#1}\frame{\partpage \tableofcontents}}
\newcommand{\mysection}[1]{\section{#1}\frame{\frametitle{Overview}\tableofcontents[sectionstyle=show/shaded,subsectionstyle=show/shaded/shaded]}}
\newcommand{\mysubsection}[1]{\subsection{#1}\frame{\frametitle{Overview}\tableofcontents[sectionstyle=show/shaded,subsectionstyle=show/shaded/shaded]}}
\newcommand{\mysubsubsection}[1]{\subsubsection{#1}\frame{\frametitle{Overview}\tableofcontents[sectionstyle=show/shaded,subsectionstyle=show/shaded/shaded]}}

\begin{document}

\psset{angleA=90,angleB=-90}
\lstset{language=Perl,
        basicstyle=\ttfamily,
        keywordstyle=\color{Maroon},
        commentstyle=\color{Blue}, 
        stringstyle=\color{Green},
        showstringspaces=false}

%\AtBeginSection[]{\begin{frame}<beamer> \frametitle{Overview} \tableofcontents[current, currentsubsection] \end{frame}}
%\AtBeginSubsection[]{\begin{frame}<beamer> \frametitle{Overview} \tableofcontents[current, currentsubsection] \end{frame}}
\AtBeginPart{\begin{frame}<beamer> \frametitle{Overview} \partpage \tableofcontents[current] \end{frame}}

\frame{\maketitle}

\part{Introduction}

\section{Introduction}

\begin{frame}[fragile]
\frametitle{Audience}
\begin{block}<1->{Audience}
\begin{itemize}
\item Developer who wants to create or improve monitoring software
\item Developer who wants to evaluate system stats for content sensitive
      code paths
\item Developer who wants to to learn the difference to earlier libstatgrab
      / Unix::Statgrab API
\item Developers or Operators (Admins) who wants to learn about measurement
      of statistic values of the machine
\end{itemize}
\end{block}

\begin{block}<2->{Prerequisites of the Audience}
Following knowledge is expected:
\begin{itemize}
\item advanced skills in at least one object oriented and procedural programming language
\item more than one year practical experience in object oriented development
\item Experience with Unix or compatible operating systems
\item slightly above basic Perl experience
\end{itemize}
\end{block}
\end{frame}

\begin{frame}[fragile]
\frametitle{Motivation}

\begin{block}<1->{XS / C}
\begin{itemize}
\item use of native API to get OS stats
\item performance advantage
\item interoperability (most VM's have a * native interface)
\item portability - widest calling convention support in C
\end{itemize}
\end{block}

\end{frame}

\begin{frame}[fragile]
\frametitle{Platforms I}
\begin{block}<1->{Tested and confirmed running}
\begin{itemize}
\item DragonFly BSD 3.4
\item FreeBSD 7,8 (i386, amd64), FreeBSD 9 (i386, amd64, sparc64, ia64), FreeBSD 10-CURRENT (i386, amd64, sparc64, ia64)
\item HP-UX 11.11 (parisc) HP-UX 11.23 (parisc, ia64), HP-UX 11.31 (ia64)
\item Linux 2.6 (Ubuntu 10.04, SLES 9-11, Redhat 6, CentOS 6, \textmu{}CLinux/arm7), Linux 3.X (Ubuntu 12.04)
\item MacOS X 10.6, 10.8 (amd64)
\item NetBSD 5.1-6.1 (amd64), NetBSD-CURRENT (amd64)
\item OpenBSD 4.9, 5.3 (amd64)
\item Solaris 8,9 (sparc), Solaris 10 (sparc, x86 \& amd64), Solaris 11 (amd64)
\item AIX 5.2, 5.3, 6.1 (ppc64)
\end{itemize}
\end{block}
\end{frame}

\begin{frame}[fragile]
\frametitle{Platforms II}
\begin{block}<1->{in progress \ldots}
\begin{itemize}
\item Windows (using Interix, maybe mSys)
\item kFreeBSD
\item Hurd
\end{itemize}
\end{block}

\begin{block}<2->{Wishlist}
\begin{itemize}
\item Digital Unix / Tru64 / OSF1
\item Haiku
\item VMS
\item zOS
\end{itemize}
\end{block}
\end{frame}

\part{libstatgrab}

\section{Host Information}

\begin{frame}[fragile]
\frametitle{Host Info}
\begin{columns}[t]
\begin{column}{.5\textwidth}
\begin{block}<1->{sg\_host\_info}
\tiny
\begin{lstlisting}[language=C++,inputencoding=latin9,escapeinside={(*@}{@*)}]
typedef struct {
    char *os_name(*@\pnode(0,0){A}{}@*);
    char *os_release(*@\pnode(0,0){B}{}@*);
    char *os_version(*@\pnode(0,0){C}{}@*);
    char *platform(*@\pnode(0,0){D}{}@*);
    char *hostname(*@\pnode(0,0){E}{}@*);
    unsigned bitwidth(*@\pnode(0,0){F}{}@*);
    sg_host_state host_state(*@\pnode(0,0){G}{}@*);
    unsigned ncpus(*@\pnode(0,0){H}{}@*);
    unsigned maxcpus(*@\pnode(0,0){I}{}@*);
    time_t uptime(*@\pnode(0,0){J}{}@*);
    time_t systime(*@\pnode(0,0){K}{}@*);
} sg_host_info;
\end{lstlisting}
\end{block}
\end{column}

\begin{column}{.5\textwidth}
\scriptsize
\uncover<2->{bundles some operating system information as}
\begin{itemize}
\onslide<2->{\item \rnode{a}{name} (Linux, FreeBSD, AIX),
             \uncover<3->{\item \rnode{b}{release} (eg. kernel version),}
             \uncover<4->{\item \rnode{c}{entire} OS version string (eg. Darwin Kernel Version 12.4.0: Wed May  1 17:57:12 PDT 2013; root:xnu-2050.24.15~1/RELEASE\_X86\_64),}
             \uncover<5->{\item \rnode{d}{platform}, what finally means CPU information from OS perspective,}
             \uncover<6->{\item \rnode{e}{hostname} name of the host.}
             \uncover<7->{\item \rnode{f}{bitwidth} (usually 32 or 64),}
             \uncover<8->{\item \rnode{g}{host state} - one of \texttt{sg\_physical\_host},
			  \texttt{sg\_virtual\_machine}, \texttt{sg\_paravirtual\_machine},
			  \texttt{sg\_hardware\_virtualized} or
			  \texttt{sg\_unknown\_configuration}),}
             \uncover<9->{\item \rnode{h}{current} number of CPU's,}
             \uncover<10->{\item \rnode{i}{maximum} number of CPU's.}
             \uncover<11->{\item \rnode{k}{timestamp} when collected this stats.}

             \uncover<2-6>{\nccurve[linecolor=olive]{->}{a}{A}}
             \uncover<3-6>{\nccurve[linecolor=olive]{->}{b}{B}}
             \uncover<4-6>{\nccurve[linecolor=olive]{->}{c}{C}}
             \uncover<5-6>{\nccurve[linecolor=olive]{->}{d}{D}}
             \uncover<6-6>{\nccurve[linecolor=olive]{->}{e}{E}}
             \uncover<7-8>{\nccurve[linecolor=olive]{->}{f}{F}}
             \uncover<8-8>{\nccurve[linecolor=olive]{->}{g}{G}}
             \uncover<9-10>{\nccurve[linecolor=olive]{->}{h}{H}}
             \uncover<10-10>{\nccurve[linecolor=olive]{->}{i}{I}}
             \uncover<11-11>{\nccurve[linecolor=olive]{->}{k}{K}}
}
\end{itemize}
\end{column}
\end{columns}
\end{frame}

\section{CPU statistics}

\begin{frame}[fragile]
\frametitle{CPU stats}
\begin{block}<1->{sg\_cpu\_stats}
\tiny
\begin{lstlisting}[language=C++,inputencoding=latin9,escapeinside={(*@}{@*)}]
typedef struct {
    unsigned long long user, kernel, idle, iowait, swap, nice, total(*@\pnode(0,0){A}{}@*);

    unsigned long long context_switches, voluntary_context_switches, involuntary_context_switches,(*@\pnode(0,0){B}{}@*)
                       syscalls, interrupts, soft_interrupts,(*@\pnode(0,0){C}{}@*)

    time_t systime;(*@\pnode(0,0){D}{}@*)
} sg_cpu_stats;
\end{lstlisting}
\end{block}

\scriptsize
\begin{itemize}
\onslide<2->{\item absolute ticks of measurable CPU states\rnode{a}{,}
             \uncover<3->{\item context switches over all CPU's, also separated by voluntary and involuntary\rnode{b}{,}}
             \uncover<4->{\item syscalls made, interrupts and soft-interrupts occured\rnode{c}{,}}
             \uncover<5->{\item timestamp when collected this stats\rnode{d}{.}}
             \uncover<2-5>{\nccurve[linecolor=olive]{->}{a}{A}}
             \uncover<3-5>{\nccurve[linecolor=olive]{->}{b}{B}}
             \uncover<4-5>{\nccurve[linecolor=olive]{->}{c}{C}}
             \uncover<5-5>{\nccurve[linecolor=olive]{->}{d}{D}}
}
\end{itemize}
\end{frame}

\begin{frame}[fragile]
\frametitle{CPU percents}
\begin{block}<1->{sg\_cpu\_percents}
\tiny
\begin{lstlisting}[language=C++,inputencoding=latin9,escapeinside={(*@}{@*)}]
typedef struct {
    double user(*@\pnode(0,0){A}{}@*);
    double kernel(*@\pnode(0,0){B}{}@*);
    double idle(*@\pnode(0,0){C}{}@*);
    double iowait(*@\pnode(0,0){D}{}@*);
    double swap(*@\pnode(0,0){E}{}@*);
    double nice(*@\pnode(0,0){F}{}@*);
    time_t time_taken(*@\pnode(0,0){G}{}@*);
} sg_cpu_percents;
\end{lstlisting}
\end{block}

\scriptsize
\begin{itemize}
\onslide<2->{\item relative ticks of measurable CPU states:
             \uncover<3->{ticks in \rnode{a}{user} mode,}
             \uncover<4->{\rnode{b}{kernel} mode,}
             \uncover<5->{\rnode{c}{idle} time,}
             \uncover<6->{waiting for \rnode{d}{i/o},}
             \uncover<7->{during page \rnode{e}{swap},}
             \uncover<8->{\rnode{f}{nice} rescheduled}
             \uncover<9->{\item timestamp when collected this stats\rnode{g}{.}}
             \uncover<3-8>{\nccurve[linecolor=olive]{->}{a}{A}}
             \uncover<4-8>{\nccurve[linecolor=olive]{->}{b}{B}}
             \uncover<5-8>{\nccurve[linecolor=olive]{->}{c}{C}}
             \uncover<6-8>{\nccurve[linecolor=olive]{->}{d}{D}}
             \uncover<7-8>{\nccurve[linecolor=olive]{->}{e}{E}}
             \uncover<8-8>{\nccurve[linecolor=olive]{->}{f}{F}}
             \uncover<9-9>{\nccurve[linecolor=olive]{->}{g}{G}}
}
\end{itemize}
\end{frame}

\begin{frame}[fragile]
\frametitle{Load percents}
\begin{block}<1->{sg\_load\_stats}
\tiny
\begin{lstlisting}[language=C++,inputencoding=latin9,escapeinside={(*@}{@*)}]
typedef struct {
    double min1(*@\pnode(0,0){A}{}@*);
    double min5(*@\pnode(0,0){B}{}@*);
    double min15(*@\pnode(0,0){C}{}@*);
    time_t systime(*@\pnode(0,0){D}{}@*);
} sg_load_stats;
\end{lstlisting}
\end{block}

\scriptsize
\begin{itemize}
\onslide<2->{\item percentage of cpu usage per
             \uncover<3->{\rnode{a}{1} minute,}
             \uncover<4->{\rnode{b}{5} minutes and}
             \uncover<5->{\rnode{c}{15} minutes.}
             \uncover<6->{\item timestamp when collected this stats\rnode{d}{.}}
             \uncover<3-5>{\nccurve[linecolor=olive]{->}{a}{A}}
             \uncover<4-5>{\nccurve[linecolor=olive]{->}{b}{B}}
             \uncover<5-5>{\nccurve[linecolor=olive]{->}{c}{C}}
             \uncover<6-6>{\nccurve[linecolor=olive]{->}{d}{D}}
}
\end{itemize}
\end{frame}

\section{Memory statistics}

\begin{frame}[fragile]
\frametitle{Memory stats}
\begin{block}<1->{sg\_mem\_stats}
\tiny
\begin{lstlisting}[language=C++,inputencoding=latin9,escapeinside={(*@}{@*)}]
typedef struct {
    unsigned long long total(*@\pnode(0,0){A}{}@*);
    unsigned long long free(*@\pnode(0,0){B}{}@*);
    unsigned long long used(*@\pnode(0,0){C}{}@*);
    unsigned long long cache(*@\pnode(0,0){D}{}@*);
    time_t systime(*@\pnode(0,0){E}{}@*);
} sg_mem_stats;
\end{lstlisting}
\end{block}

\scriptsize
\begin{itemize}
\onslide<2->{\item information about main memory of the system:
             \uncover<3->{\rnode{a}{total},}
             \uncover<4->{\rnode{b}{free},}
             \uncover<5->{\rnode{c}{used},}
             \uncover<6->{\rnode{d}{cache},}
             \uncover<7->{\item timestamp when collected this stats\rnode{e}{.}}
             \uncover<3-6>{\nccurve[linecolor=olive]{->}{a}{A}}
             \uncover<4-6>{\nccurve[linecolor=olive]{->}{b}{B}}
             \uncover<5-6>{\nccurve[linecolor=olive]{->}{c}{C}}
             \uncover<6-6>{\nccurve[linecolor=olive]{->}{d}{D}}
             \uncover<7-7>{\nccurve[linecolor=olive]{->}{e}{E}}
}
\end{itemize}
\end{frame}

\begin{frame}[fragile]
\frametitle{Swap stats}
\begin{block}<1->{sg\_swap\_stats}
\tiny
\begin{lstlisting}[language=C++,inputencoding=latin9,escapeinside={(*@}{@*)}]
typedef struct {
    unsigned long long total(*@\pnode(0,0){A}{}@*);
    unsigned long long free(*@\pnode(0,0){B}{}@*);
    unsigned long long used(*@\pnode(0,0){C}{}@*);
    time_t systime(*@\pnode(0,0){D}{}@*);
} sg_swap_stats;
\end{lstlisting}
\end{block}

\scriptsize
\begin{itemize}
\onslide<2->{\item information about swap memory of the system:
             \uncover<3->{\rnode{a}{total},}
             \uncover<4->{\rnode{b}{free},}
             \uncover<5->{\rnode{c}{used},}
             \uncover<6->{\item timestamp when collected this stats\rnode{d}{.}}
             \uncover<3-5>{\nccurve[linecolor=olive]{->}{a}{A}}
             \uncover<4-5>{\nccurve[linecolor=olive]{->}{b}{B}}
             \uncover<5-5>{\nccurve[linecolor=olive]{->}{c}{C}}
             \uncover<6-6>{\nccurve[linecolor=olive]{->}{d}{D}}
}
\end{itemize}
\end{frame}

\section{Disk / Storage statistics}

\begin{frame}[fragile]
\frametitle{Disk I/O stats}
\begin{block}<1->{sg\_disk\_io\_stats}
\tiny
\begin{lstlisting}[language=C++,inputencoding=latin9,escapeinside={(*@}{@*)}]
typedef struct {
    char *disk_name(*@\pnode(0,0){A}{}@*);
    unsigned long long read_bytes(*@\pnode(0,0){B}{}@*);
    unsigned long long write_bytes(*@\pnode(0,0){C}{}@*);
    time_t systime(*@\pnode(0,0){D}{}@*);
} sg_disk_io_stats;
\end{lstlisting}
\end{block}

\scriptsize
\uncover<2->{for each block device known to the system}
\begin{itemize}
\onslide<2->{\item \rnode{a}{name} of the block device,
             \uncover<3->{\item amount of bytes \rnode{b}{read}}
             \uncover<4->{\item amount of bytes \rnode{c}{written}}
             \uncover<5->{\item timestamp when collected this stats\rnode{d}{.}}
             \uncover<2-2>{\nccurve[linecolor=olive]{->}{a}{A}}
             \uncover<3-4>{\nccurve[linecolor=olive]{->}{b}{B}}
             \uncover<4-4>{\nccurve[linecolor=olive]{->}{c}{C}}
             \uncover<5-5>{\nccurve[linecolor=olive]{->}{d}{D}}
}
\end{itemize}
\end{frame}

\begin{frame}[fragile]
\frametitle{Paging stats}
\begin{block}<1->{sg\_page\_stats}
\tiny
\begin{lstlisting}[language=C++,inputencoding=latin9,escapeinside={(*@}{@*)}]
typedef struct {
    unsigned long long pages_pagein(*@\pnode(0,0){A}{}@*);
    unsigned long long pages_pageout(*@\pnode(0,0){B}{}@*);
    time_t systime(*@\pnode(0,0){C}{}@*);
} sg_page_stats;
\end{lstlisting}
\end{block}

\scriptsize
\uncover<2->{for entire system}
\begin{itemize}
\onslide<2->{\item amount of bytes \rnode{a}{paged in},
             \uncover<3->{\item amount of bytes \rnode{b}{paged out}}
             \uncover<4->{\item timestamp when collected this stats\rnode{c}{.}}
             \uncover<2-3>{\nccurve[linecolor=olive]{->}{a}{A}}
             \uncover<3-3>{\nccurve[linecolor=olive]{->}{b}{B}}
             \uncover<4-4>{\nccurve[linecolor=olive]{->}{c}{C}}
}
\end{itemize}
\end{frame}

\begin{frame}[fragile]
\frametitle{Filesystem stats}
\begin{columns}[t]
\begin{column}{.5\textwidth}
\begin{block}<1->{sg\_fs\_stats}
\tiny
\begin{lstlisting}[language=C++,inputencoding=latin9,escapeinside={(*@}{@*)}]
typedef struct {
    char *device_name(*@\pnode(0,0){A}{}@*);
    char *fs_type(*@\pnode(0,0){B}{}@*);
    char *mnt_point(*@\pnode(0,0){C}{}@*);
    sg_fs_device_type device_type(*@\pnode(0,0){D}{}@*);
    unsigned long long size(*@\pnode(0,0){E}{}@*);
    unsigned long long used(*@\pnode(0,0){F}{}@*);
    unsigned long long free(*@\pnode(0,0){G}{}@*);
    unsigned long long avail(*@\pnode(0,0){H}{}@*);
    unsigned long long total_inodes(*@\pnode(0,0){I}{}@*);
    unsigned long long used_inodes(*@\pnode(0,0){J}{}@*);
    unsigned long long free_inodes(*@\pnode(0,0){K}{}@*);
    unsigned long long avail_inodes(*@\pnode(0,0){L}{}@*);
    unsigned long long io_size(*@\pnode(0,0){M}{}@*);
    unsigned long long block_size(*@\pnode(0,0){N}{}@*);
    unsigned long long total_blocks(*@\pnode(0,0){O}{}@*);
    unsigned long long free_blocks(*@\pnode(0,0){P}{}@*);
    unsigned long long used_blocks(*@\pnode(0,0){Q}{}@*);
    unsigned long long avail_blocks(*@\pnode(0,0){R}{}@*);
    time_t systime(*@\pnode(0,0){S}{}@*);
} sg_fs_stats;
\end{lstlisting}
\end{block}
\end{column}

\begin{column}{.5\textwidth}
\scriptsize
\uncover<2->{for each mounted (and not filtered) file system}
\begin{itemize}
\onslide<2->{\item \rnode{a}{name} of the mounted block device,
             \uncover<3->{\item name of the \rnode{b}{file system} type (eg. ext3, ffs, zfs)}
             \uncover<4->{\item full qualified path name of the \rnode{c}{mount point}}
             \uncover<5->{\item \rnode{d}{device type}: one of \texttt{sg\_fs\_unknown}, \texttt{sg\_fs\_regular},
			  \texttt{sg\_fs\_special}, \texttt{sg\_fs\_loopback}, \texttt{sg\_fs\_remote} or any
			  combination\\
			  Anything but unknown is covered by \texttt{sg\_fs\_alltypes}, any local type by
			  \texttt{sg\_fs\_local}}
             \uncover<6->{\item \rnode{e}{size} of the file system in bytes}
             \uncover<7->{\item also separated into \rnode{f}{used}, \rnode{g}{free} and \rnode{h}{avail}}
             \uncover<8->{\item \rnode{i}{inodes} of the file system}
             \uncover<9->{\item also separated into \rnode{j}{used}, \rnode{k}{free} and \rnode{l}{avail}}
             \uncover<10->{\item optimal \rnode{m}{size} of the I/O blocks when accessing the file system in bytes}
             \uncover<11->{\item \rnode{n}{block size} (minimum allocation size) of the file system in bytes}
             \uncover<12->{\item \rnode{o}{amount} of blocks of the file system}
             \uncover<13->{\item also separated into \rnode{p}{used}, \rnode{q}{free} and \rnode{r}{avail}}
             \uncover<14->{\item timestamp when collected this stats\rnode{s}{.}}
             \uncover<2-4>{\nccurve[linecolor=olive]{->}{a}{A}}
             \uncover<3-4>{\nccurve[linecolor=olive]{->}{b}{B}}
             \uncover<4-4>{\nccurve[linecolor=olive]{->}{c}{C}}
             \uncover<5-5>{\nccurve[linecolor=olive]{->}{d}{D}}
             \uncover<6-7>{\nccurve[linecolor=olive]{->}{e}{E}}
             \uncover<7-7>{\nccurve[linecolor=olive]{->}{f}{F}}
             \uncover<7-7>{\nccurve[linecolor=olive]{->}{g}{G}}
             \uncover<7-7>{\nccurve[linecolor=olive]{->}{h}{H}}
             \uncover<8-9>{\nccurve[linecolor=olive]{->}{i}{I}}
             \uncover<9-9>{\nccurve[linecolor=olive]{->}{j}{J}}
             \uncover<9-9>{\nccurve[linecolor=olive]{->}{k}{K}}
             \uncover<9-9>{\nccurve[linecolor=olive]{->}{l}{L}}
             \uncover<10-13>{\nccurve[linecolor=olive]{->}{m}{M}}
             \uncover<11-13>{\nccurve[linecolor=olive]{->}{n}{N}}
             \uncover<12-13>{\nccurve[linecolor=olive]{->}{o}{O}}
             \uncover<13-13>{\nccurve[linecolor=olive]{->}{p}{P}}
             \uncover<13-13>{\nccurve[linecolor=olive]{->}{q}{Q}}
             \uncover<13-13>{\nccurve[linecolor=olive]{->}{r}{R}}
             \uncover<14-14>{\nccurve[linecolor=olive]{->}{s}{S}}
}
\end{itemize}
\end{column}
\end{columns}
\end{frame}

\section{User statistics}

\begin{frame}[fragile]
\frametitle{User stats}
\begin{block}<1->{sg\_user\_stats}
\tiny
\begin{lstlisting}[language=C++,inputencoding=latin9,escapeinside={(*@}{@*)}]
typedef struct {
    char *login_name(*@\pnode(0,0){A}{}@*);
    char *record_id(*@\pnode(0,0){B}{}@*);
    size_t record_id_size(*@\pnode(0,0){C}{}@*);
    char *device(*@\pnode(0,0){D}{}@*);
    char *hostname(*@\pnode(0,0){E}{}@*);
    pid_t pid(*@\pnode(0,0){F}{}@*);
    time_t login_time(*@\pnode(0,0){G}{}@*);
    time_t systime(*@\pnode(0,0){H}{}@*);
} sg_user_stats;
\end{lstlisting}
\end{block}

\scriptsize
\uncover<2->{statistics about logged in users, as}
\begin{itemize}
\onslide<2->{\item \rnode{a}{login name},
             \uncover<3->{\item \rnode{b}{record id} and \rnode{c}{size} of that field (not \texttt{'\textbackslash{}0'} terminated),}
             \uncover<4->{\item \rnode{d}{device} where user logged in,}
             \uncover<5->{\item \rnode{e}{hostname} when remote login}
             \uncover<6->{\item \rnode{f}{process id} of the session's ''root'' process}
             \uncover<7->{\item \rnode{g}{login time} of that session}
             \uncover<8->{\item timestamp when collected this stats\rnode{h}{.}}
             \uncover<2-2>{\nccurve[linecolor=olive]{->}{a}{A}}
             \uncover<3-4>{\nccurve[linecolor=olive]{->}{b}{B}}
             \uncover<3-4>{\nccurve[linecolor=olive]{->}{c}{C}}
             \uncover<4-5>{\nccurve[linecolor=olive]{->}{d}{D}}
             \uncover<5-5>{\nccurve[linecolor=olive]{->}{e}{E}}
             \uncover<6-6>{\nccurve[linecolor=olive]{->}{f}{F}}
             \uncover<7-7>{\nccurve[linecolor=olive]{->}{g}{G}}
             \uncover<8-8>{\nccurve[linecolor=olive]{->}{h}{H}}
}
\end{itemize}
\end{frame}

\section{Process statistics}

\begin{frame}[fragile]
\frametitle{Process stats}
\begin{columns}[t]
\begin{column}{.5\textwidth}
\begin{block}<1->{sg\_process\_stats}
\tiny
\begin{lstlisting}[language=C++,inputencoding=latin9,escapeinside={(*@}{@*)}]
typedef struct {
    char *process_name(*@\pnode(0,0){A}{}@*);
    char *proctitle(*@\pnode(0,0){B}{}@*);
    pid_t pid(*@\pnode(0,0){C}{}@*);
    pid_t parent(*@\pnode(0,0){D}{}@*);
    pid_t pgid(*@\pnode(0,0){E}{}@*);
    pid_t sessid(*@\pnode(0,0){F}{}@*);
    uid_t uid(*@\pnode(0,0){G}{}@*);
    uid_t euid(*@\pnode(0,0){H}{}@*);
    gid_t gid(*@\pnode(0,0){I}{}@*);
    gid_t egid(*@\pnode(0,0){J}{}@*);
    unsigned long long context_switches(*@\pnode(0,0){K}{}@*);
    unsigned long long voluntary_context_switches(*@\pnode(0,0){L}{}@*);
    unsigned long long involuntary_context_switches(*@\pnode(0,0){M}{}@*);
    unsigned long long proc_size(*@\pnode(0,0){N}{}@*);
    unsigned long long proc_resident(*@\pnode(0,0){O}{}@*);
    time_t start_time(*@\pnode(0,0){P}{}@*);
    time_t time_spent(*@\pnode(0,0){Q}{}@*);
    double cpu_percent(*@\pnode(0,0){R}{}@*);
    int nice(*@\pnode(0,0){S}{}@*);
    sg_process_state state(*@\pnode(0,0){T}{}@*);
    time_t systime(*@\pnode(0,0){U}{}@*);
} sg_process_stats;
\end{lstlisting}
\end{block}
\end{column}

\begin{column}{.5\textwidth}
\scriptsize
\uncover<2->{for each existing process}
\begin{itemize}
\onslide<2->{\item name of the \rnode{a}{process image},
             \uncover<3->{\item title of the \rnode{b}{process} (usually FQPN + args)}
             \uncover<4->{\item process id of the \rnode{c}{process},}
             \uncover<5->{the \rnode{d}{parent process},}
             \uncover<6->{the \rnode{e}{process group leader} and}
             \uncover<7->{the \rnode{f}{session id} of the session the process belongs to}
             \uncover<8->{\item process' \rnode{g}{user id},}
             \uncover<9->{\rnode{h}{group id},}
             \uncover<10->{\rnode{i}{effective user id} and}
             \uncover<11->{\rnode{j}{effective group id}}
             \uncover<12->{\item \rnode{k}{context switches} done by the process, also separated by \rnode{l}{voluntary} and \rnode{m}{involuntary}}
             \uncover<13->{\item \rnode{n}{virtual} memory size of the process, thereof \rnode{O}{resident}}
             \uncover<14->{\item \rnode{p}{start time} of the process, \rnode{q}{time spent} on CPU during lifetime, \rnode{r}{relative} to system usage}
             \uncover<15->{\item \rnode{s}{nice} value of the process (process scheduling increment)}
             \uncover<16->{\item \rnode{t}{device type}: one of \texttt{SG\_PROCESS\_STATE\_RUNNING},
			  \texttt{SG\_PROCESS\_STATE\_SLEEPING}, \texttt{SG\_PROCESS\_STATE\_STOPPED},
			  \texttt{SG\_PROCESS\_STATE\_ZOMBIE} or \texttt{SG\_PROCESS\_STATE\_UNKNOWN}}
             \uncover<17->{\item timestamp when collected this stats\rnode{u}{.}}
             \uncover<2-3>{\nccurve[linecolor=olive]{->}{a}{A}}
             \uncover<3-3>{\nccurve[linecolor=olive]{->}{b}{B}}
             \uncover<4-7>{\nccurve[linecolor=olive]{->}{c}{C}}
             \uncover<5-7>{\nccurve[linecolor=olive]{->}{d}{D}}
             \uncover<6-7>{\nccurve[linecolor=olive]{->}{e}{E}}
             \uncover<7-7>{\nccurve[linecolor=olive]{->}{f}{F}}
             \uncover<8-11>{\nccurve[linecolor=olive]{->}{g}{G}}
             \uncover<9-11>{\nccurve[linecolor=olive]{->}{h}{H}}
             \uncover<10-11>{\nccurve[linecolor=olive]{->}{i}{I}}
             \uncover<11-11>{\nccurve[linecolor=olive]{->}{j}{J}}
             \uncover<12-12>{\nccurve[linecolor=olive]{->}{k}{K}\nccurve[linecolor=olive]{->}{l}{L}\nccurve[linecolor=olive]{->}{m}{M}}
             \uncover<13-13>{\nccurve[linecolor=olive]{->}{n}{N}\nccurve[linecolor=olive]{->}{o}{O}}
             \uncover<14-14>{\nccurve[linecolor=olive]{->}{p}{P}\nccurve[linecolor=olive]{->}{q}{Q}\nccurve[linecolor=olive]{->}{r}{R}}
             \uncover<15-15>{\nccurve[linecolor=olive]{->}{s}{S}}
             \uncover<16-16>{\nccurve[linecolor=olive]{->}{t}{T}}
             \uncover<17-17>{\nccurve[linecolor=olive]{->}{u}{U}}
}
\end{itemize}
\end{column}
\end{columns}
\end{frame}

\section{Network statistics}

\begin{frame}[fragile]
\frametitle{Network I/O stats}
\begin{block}<1->{sg\_network\_io\_stats}
\tiny
\begin{lstlisting}[language=C++,inputencoding=latin9,escapeinside={(*@}{@*)}]
typedef struct {
    char *interface_name(*@\pnode(0,0){A}{}@*);
    unsigned long long (*@\pnode(0,0){B}{}@*)tx;
    unsigned long long rx(*@\pnode(0,0){C}{}@*);
    unsigned long long (*@\pnode(0,0){D}{}@*)ipackets;
    unsigned long long opackets(*@\pnode(0,0){E}{}@*);
    unsigned long long (*@\pnode(0,0){F}{}@*)ierrors;
    unsigned long long oerrors(*@\pnode(0,0){G}{}@*);
    unsigned long long collisions(*@\pnode(0,0){H}{}@*);
    time_t systime(*@\pnode(0,0){I}{}@*);
} sg_network_io_stats;
\end{lstlisting}
\end{block}

\scriptsize
\uncover<2->{for each network interface}
\begin{itemize}
\onslide<2->{\item \rnode{a}{interface name} (eg. ''em0'', ''fxp0'', ''en0'', ''eth0'' \ldots),
             \uncover<3->{\item bytes \rnode{b}{transmitted} and \rnode{c}{received}}
             \uncover<4->{\item packets \rnode{d}{transmitted} and \rnode{e}{received}}
             \uncover<5->{\item errors \rnode{f}{transmitting} and \rnode{g}{receiving} packets}
             \uncover<6->{\item detected \rnode{h}{collisions}}
             \uncover<7->{\item timestamp when collected this stats\rnode{i}{.}}
             \uncover<2-2>{\nccurve[linecolor=olive]{->}{a}{A}}
             \uncover<3-3>{\nccurve[linecolor=olive]{->}{b}{B}\nccurve[linecolor=olive]{->}{c}{C}}
             \uncover<4-4>{\nccurve[linecolor=olive]{->}{d}{D}\nccurve[linecolor=olive]{->}{e}{E}}
             \uncover<5-6>{\nccurve[linecolor=olive]{->}{f}{F}\nccurve[linecolor=olive]{->}{g}{G}}
             \uncover<6-6>{\nccurve[linecolor=olive]{->}{h}{H}}
             \uncover<7-7>{\nccurve[linecolor=olive]{->}{i}{I}}
}
\end{itemize}
\end{frame}

\begin{frame}[fragile]
\frametitle{Network Interface stats}
\begin{block}<1->{sg\_network\_iface\_stats}
\tiny
\begin{lstlisting}[language=C++,inputencoding=latin9,escapeinside={(*@}{@*)}]
typedef struct {
    char *interface_name(*@\pnode(0,0){A}{}@*);
    unsigned long long speed(*@\pnode(0,0){B}{}@*);
    unsigned long long factor(*@\pnode(0,0){C}{}@*);
    sg_iface_duplex duplex(*@\pnode(0,0){D}{}@*);
    sg_iface_updown up(*@\pnode(0,0){E}{}@*);
    time_t systime(*@\pnode(0,0){F}{}@*);
} sg_network_iface_stats;
\end{lstlisting}
\end{block}

\scriptsize
\uncover<2->{for each network interface}
\begin{itemize}
\onslide<2->{\item \rnode{a}{interface name} (eg. ''em0'', ''fxp0'', ''en0'', ''eth0'' \ldots),
             \uncover<3->{\item capable to transfer \rnode{b}{times} of \rnode{c}{sized} units per second}
             \uncover<4->{\item capable to transmit and receive \rnode{d}{simultanously} (\texttt{SG\_IFACE\_DUPLEX\_FULL}, \texttt{SG\_IFACE\_DUPLEX\_HALF} or \texttt{SG\_IFACE\_DUPLEX\_UNKNOWN})}
             \uncover<5->{\item NIC is \rnode{e}{\texttt{SG\_IFACE\_UP} or \texttt{SG\_IFACE\_DOWN}}}
             \uncover<6->{\item timestamp when collected this stats\rnode{f}{.}}
             \uncover<2-2>{\nccurve[linecolor=olive]{->}{a}{A}}
             \uncover<3-3>{\nccurve[linecolor=olive]{->}{b}{B}\nccurve[linecolor=olive]{->}{c}{C}}
             \uncover<4-4>{\nccurve[linecolor=olive]{->}{d}{D}}
             \uncover<5-5>{\nccurve[linecolor=olive]{->}{e}{E}}
             \uncover<6-6>{\nccurve[linecolor=olive]{->}{f}{F}}
}
\end{itemize}
\end{frame}

\section{Error management}

\begin{frame}[fragile]
\frametitle{Error information}
\begin{block}<1->{sg\_error\_details}
\tiny
\begin{lstlisting}[language=C++,inputencoding=latin9,escapeinside={(*@}{@*)}]
typedef struct sg_error_details {
	sg_error error(*@\pnode(0,0){A}{}@*);
	int errno_value(*@\pnode(0,0){B}{}@*);
	const char *error_arg(*@\pnode(0,0){C}{}@*);
} sg_error_details;
\end{lstlisting}
\end{block}

\scriptsize
\uncover<2->{when an error occured (no stats are resulted upon querying):}
\begin{itemize}
\onslide<2->{\item libstatgrab \rnode{a}{error code} (eg. \texttt{SG\_ERROR\_INVALID\_ARGUMENT})
             \uncover<3->{\item system (errno.h) \rnode{b}{error code} (eg. \texttt{EBUSY}}
             \uncover<4->{\item optional error \rnode{c}{explanation} message (eg. file name, process id, \ldots)}
             \uncover<2-2>{\nccurve[linecolor=olive]{->}{a}{A}}
             \uncover<3-3>{\nccurve[linecolor=olive]{->}{b}{B}}
             \uncover<4-4>{\nccurve[linecolor=olive]{->}{c}{C}}
}
\end{itemize}
\end{frame}

\part{Unix::Statgrab}

\section{Entry Functions}

\begin{frame}[fragile]
\frametitle{Entry Functions}
\begin{block}<1->{}
\tiny
\begin{lstlisting}[language=Perl,inputencoding=latin9,escapeinside={(*@}{@*)}]
get_error(); # return details about last error
get_host_info(); # returns sg_host_info
get_cpu_stats(); # returns sg_cpu_stats
get_disk_io_stats(); # returns sg_disk_io_stats
get_fs_stats(); # returns sg_fs_stats
get_load_stats(); # returns sg_load_stats
get_mem_stats(); # returns sg_mem_stats
get_swap_stats(); # returns sg_swap_stats
get_network_io_stats(); # returns sg_network_io_stats
get_network_iface_stats(); # returns sg_network_iface_stats
get_page_stats(); # returns sg_page_stats
get_user_stats(); # returns sg_user_stats
get_process_stats(); # returns sg_process_stats
\end{lstlisting}
\end{block}
\end{frame}

\section{SYNOPSIS}

\begin{frame}[fragile]
\frametitle{Common \ldots}
\begin{block}<1->{Common \ldots}
\tiny
\begin{lstlisting}[language=Perl,inputencoding=latin9,escapeinside={(*@}{@*)}]
use Unix::Statgrab;

my $host_info = get_host_info() or croak( get_error()->strperror() );
printf( "%d\n", $host_info->entries() );
my $cpu_stats = get_cpu_stats() or croak( get_error()->strperror() );
printf( "%d\n", $cpu_stats->entries() );
my $disk_io_stats = get_disk_io_stats() or croak( get_error()->strperror() );
printf( "%d\n", $disk_io_stats->entries() );
my $fs_stats = get_fs_stats() or croak( get_error()->strperror() );
printf( "%d\n", $fs_stats->entries() );
my $load_stats = get_load_stats() or croak( get_error()->strperror() );
printf( "%d\n", $load_stats->entries() );
my $mem_stats = get_mem_stats() or croak( get_error()->strperror() );
printf( "%d\n", $mem_stats->entries() );
my $swap_stats = get_swap_stats() or croak( get_error()->strperror() );
printf( "%d\n", $swap_stats->entries() );
my $net_io_stats = get_network_io_stats() or croak( get_error()->strperror() );
printf( "%d\n", $net_io_stats->entries() );
my $net_iface_stats = get_network_iface_stats() or croak( get_error()->strperror() );
printf( "%d\n", $net_iface_stats->entries() );
my $paging_stats = get_page_stats() or croak( get_error()->strperror() );
printf( "%d\n", $paging->entries() );
my $user_stats = get_user_stats() or croak( get_error()->strperror() );
printf( "%d\n", $user_stats->entries() );
my $proc_stats = get_process_stats() or croak( get_error()->strperror() );
printf( "%d\n", $proc_stats->entries() );
\end{lstlisting}
\end{block}
\end{frame}

\begin{frame}[fragile]
\frametitle{SYNOPSIS}
\begin{block}<1->{SYNOPSIS}
\tiny
\begin{lstlisting}[language=Perl,inputencoding=latin9,escapeinside={(*@}{@*)}]
use Unix::Statgrab;

my $host_stats = get_host_info();
print $host_stats->hostname . " is a " . $host_stats->bitwidth . " " . $host_stats->os_name . "\n";

my $filesystems = get_fs_stats();
my @mount_points = map { $filesystems->mnt_point($_) } (0 .. $filesystems->entries() - 1);
print $host_stats->hostname . " has " . join( ", ", @mount_points ) . " mounted\n";

my $proc_list = get_process_stats();
my @proc_by_type;
foreach my $proc_entry (0 .. $proc_list->entries() - 1) {
    $proc_by_type[$proc_list->state($proc_entry)]++;
}
my $total_procs = 0;
$total_procs += $_ for grep { defined $_ } @proc_by_type;
foreach my $state (qw(SG_PROCESS_STATE_RUNNING SG_PROCESS_STATE_SLEEPING
                      SG_PROCESS_STATE_STOPPED SG_PROCESS_STATE_ZOMBIE
		      		      SG_PROCESS_STATE_UNKNOWN)) {
    defined $proc_by_type[Unix::Statgrab->$state] or next;
    print $proc_by_type[Unix::Statgrab->$state] . " of " . $total_procs . " procs in $state\n";
}
\end{lstlisting}
\end{block}
\end{frame}

% SYNOPSIS diff / percent
\begin{frame}[fragile]
\frametitle{SYNOPSIS II}
\begin{block}<1->{SYNOPSIS diff / percent}
\tiny
\begin{lstlisting}[language=Perl,inputencoding=latin9,escapeinside={(*@}{@*)}]
use Unix::Statgrab;

my $last_cpu_stats = get_cpu_stats() or croak( get_error()->strperror() );
do_sth_way_longer();
my $cpu_diff = get_cpu_stats()->get_cpu_stats_diff($last_cpu_stats);

my $last_cpu_percent = $last_cpu_percent->get_cpu_percents();
my $diff_cpu_percent = $cpu_diff->get_cpu_percents();
my $now_cpu_percent = get_cpu_stats()->get_cpu_percents();

my $last_disk_io_stats = get_disk_io_stats() or croak( get_error()->strperror() );
do_sth_way_longer();
my $disk_io_diff = get_disk_io_stats()->get_disk_io_stats_diff($last_disk_io_stats);

my $last_fs_stats = get_fs_stats() or croak( get_error()->strperror() );
do_sth_way_longer();
my $fs_diff = get_fs_stats()->get_fs_stats_diff($last_fs_stats);

my $last_net_io_stats = get_network_io_stats() or croak( get_error()->strperror() );
do_sth_way_longer();
my $net_io_diff = get_network_io_stats()->get_network_io_stats_diff($last_net_io_stats);

my $last_paging_stats = get_page_stats() or croak( get_error()->strperror() );
do_sth_way_longer();
my $paging_diff = get_page_stats()->get_page_stats_diff($last_paging_stats);
\end{lstlisting}
\end{block}
\end{frame}


% RESOURCES
\begin{frame}[fragile]
\frametitle{Resources}
\begin{block}<1->{Software}
\url{http://www.i-scream.org/libstatgrab/} \\
\url{http://search.cpan.org/dist/Unix-Statgrab/} \\
\url{https://metacpan.org/module/Unix::Statgrab}
\end{block}

\begin{block}<2->{Mailing List}
\url{https://lists.i-scream.org/pipermail/users/} \\
\url{users@i-scream.org}
\end{block}

\begin{block}<3->{IRC}
\url{irc://irc.freenode.net/#libstatgrab}
\end{block}
\end{frame}

% THANKS
\begin{frame}[fragile]
\frametitle{Thank You}
\begin{block}<1->{Thank you}
\begin{itemize}
\uncover<2->{\item Tim Bishop for caring for high quality release}
\uncover<3->{\item H. Merijn Brand for doing additional tests on more exotic platforms}
\uncover<4->{\item Reini Urban for proving on commodity hardware for being sane}
\end{itemize}
\end{block}
\begin{block}<5->{Questions?}
Jens Rehsack \textless{}\href{mailto:rehsack@cpan.org}{rehsack@cpan.org}\textgreater{}
\end{block}
\end{frame}

%\begin{frame}[fragile]
%\frametitle{}
%\begin{block}<1->{}
%\begin{itemize}
%\item 
%\end{itemize}
%\end{block}
%\end{frame}

\end{document}
